\section{A background on \ac{WCA}}\label{sec:background}

In~\cite{olguinmunoz:impact2021} we studied the effects of system responsiveness on human behavior in step-based \ac{WCA} through human-subject trials.
We employed a modified and instrumented version of the LEGO Assistant application devised by~\textcite{Chen2015LEGO}, which belongs to a category of \acp{WCA} with the goal of guiding a user through a sequential task.
\todo[inline]{Talk more generally about step-based-WCA, time-to-feedback in more general sense.}
\todo[inline]{WCA tracks progress autonoumosly, provides feedback.}
\todo[inline]{Token based flow, zero-wait sampling, discarded samples.}
The application monitors the progress of the task in ``realtime'' by repeatedly sampling the state of the physical system, most commonly through video frames.
Whenever the applications detects that the user has correctly or incorrectly performed an instruction, it provides a new instruction to either advance the task or correct the detected mistake.
Subjects interacted with this \ac{WCA} while the responsiveness of the system was altered in realtime and we captured key application and task performance metrics.
Additionally, we also employed questionnaires to evaluate key personality traits of the participants, and correlated these results to the task performance metrics collected.

From this experimentation, we identified four main results concerning the step execution times of humans in \ac{WCA}:

\begin{enumerate}
    \item System slow-down induces \emph{additional} behavioral slow-down which scales with the decrease in system responsiveness.
    Moreover, users become progressively slower the longer they spend in a degraded system state.

    \item The effects of system slow-down on human behavior remain for a while even after system responsiveness improves.
    
    \item Humans get faster at performing steps as the task progresses. However, this effect is dampened by reduced system responsiveness, even disappearing at higher levels of system impairment.
    
    \item The above effects seem to be modulated by the Big Five~\cite{oliver:bfi1999} trait of \emph{neuroticism}.
\end{enumerate}
\todo[inline]{Needs more precision.}

In the following, we employ the above insights together with the actual data collected for~\cite{olguinmunoz:impact2021} to design and build a probabilistic model of human behavior for \ac{WCA}.
In order to accurately emulate the behavior of a human, such a model for needs to implement two main behaviors.
One, it needs to generate realistic \emph{execution times} for each step in the task, considering the current and historical impairment of the \ac{WCA} system.
And two, it needs to produce sequences of video frames for each step which mimic what a human generate.

\subsection{Formal definitions}
\todo[inline]{Reorganize to beginning of section}

We begin by providing some definitions relating to step-based \acl{WCA}.
First of all, a \emph{step} is understood as a specific action to be performed by the user, described by a single instruction; a \emph{task} consists of a series of steps to be executed in sequence (see \cref{fig:task}).
More formally, a step starts when the corresponding instruction is provided to the user, and ends when the instruction for the next step is provided.
We call the time interval between these two events the \emph{step duration}.

Take \( \{ t_0, t_1, \ldots, t_{n + 1} \} \) a series of discrete and sequential sampling intervals at which the \ac{WCA} captures the state of the physical system --- see \cref{fig:step}.
\( t_0 \) corresponds to the instant at which the instruction for step \( M \) is provided and the first sample for said step is taken, and \( t_{n+1} \) to the instant at which the instruction for step \( M + 1 \) is provided.
Furthermore, \( t_c \) represents the instant at which the user finishes performing the instruction for step \( M \); the interval \( t_c - t_0 \) we call the \emph{execution time} of \( M \).
From the discrete sampling nature of the \ac{WCA} application, it follows that (\( U(a, b) \) represents the continuous uniform distribution in the open interval \( (a, b) \)) 


\begin{align}\label{eq:tc}
    t_c &\in U(t_{n - 1}, t_n)
\end{align}

In plain English, the user will always finish the instruction at some instant \( t_c \) which can be assumed to be randomly distributed between the next-to-last sampling instant \( t_{n - 1} \) and the last sample of a step, \( t_n \).

We further define the intervals \( t_n - t_c \) and \( t_{n + 1} - t_n \) as the \emph{wait time} and \emph{last sample \ac{RTT}}, respectively, for step \( M \).
The sum of these two values (i.e. the interval \( t_{n + 1} - t_c \)) we call \emph{time-to-feedback}.

In~\cite{olguinmunoz:impact2021} we used \emph{delay} and \emph{execution time} as our key variables of interest.
\emph{Delay} corresponded to a set time duration the processing of each frame was padded to --- i.e. if during a series of steps delay was set to \( X \) seconds, the feedback for each frame was provided to the user at exactly \( X \) seconds after frame capture.
The correlation between delay and execution time was then studied.
Delay, however, was merely an experimental tool, and in the present paper we argue that the underlying metric driving  human behavior is actually \emph{time-to-feedback}.
If we consider that
\begin{enumerate*}[itemjoin={{; }}, itemjoin={{; and }}]
    \item \( t_c \) is uniformly distributed in the interval \( [t_{n - 1}, t_{n}] \) (see~~\cite{olguinmunoz:impact2021} and \cref{eq:tc} above)
    \item for a step subject to \emph{delay} \( X \), \( t_k - t_{k - 1} = X \) for all \( k \in \{1, 2, \ldots, n\} \)
\end{enumerate*},
then it must follow that, \emph{on average}, for all steps subject to a \emph{delay} \( X \):

\begin{equation}\label{eq:ttf}
    \text{\emph{time-to-feedback}} = \frac{3}{2}X
\end{equation}

Ergo, we can directly re-parameterize the data from~\cite{olguinmunoz:impact2021} to use time-to-feedback instead of delay by simply multiplying the delay value of each step by \( 1.5 \) without loss of accuracy.

\newpage