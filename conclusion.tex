\section{Conclusion}\label{sec:conclusion}

The study and benchmarking of \ac{WCA} applications is a challenging discipline due to these application's intrinsic human-in-the-loop nature.
Humans are notoriously unreliable, and greatly complicate the scalability and repeatability of experiments.
Furthermore, recruiting large enough cohorts of humans for large-scale experimentation is both greatly time-consuming and prohibitively expensive for many research groups.

In the first half of this paper, we have introduced the \edgedroid{} model of human timing behavior for \ac{WCA}, the first data-driven model for human timings in \ac{WCA} applications.
This model represents a stochastic approach to execution time modeling which builds upon the data collected for \textcite{olguinmunoz:impact2021}.
Together with this model, we have also introduced a novel procedure for the generation of synthetic traces of frames in step-based \ac{WCA}, allowing for a full end-to-end emulation of a human when combined with the timing model.

In the second half, we have explored the potential for optimization in \ac{WCA} systems using the previously discussed timing models.
We have proposed a novel stochastic optimization framework for resource consumption-system responsiveness trade-offs in \ac{WCA}.
We have shown that this framework is applicable to a myriad of parameters in these applications, and showcased experimental results employing this framework for the minimization of number of samples processed per step and total energy consumption per step.
Our results show up to a \SI{50}{\percent} increase in performance with respect to state of the art when optimizing for number of samples, and up to a \SI{30}{\percent} improvement when optimizing for energy consumption, proving thus the value of such frameworks for the design of \ac{WCA} applications.

\todo[inline]{Finish}