\appendix{}\label[appendixwithoutnumber]{appx1}
% \subsection*{Sampling optimisation}

In this appendix, we give the mathematical derivation of the optimum aperiodic sampling interval we discussed in section \cref{ssec:optframework}.
We start with the general solution, where the objective function to be minimized is given by \cref{eq:epsilon_terminal}:
\begin{alignat}{1}
    \Rightarrow\mathcal{E}&=\alpha\mathbb{E}[\mathcal{S}]+\beta\mathbb{E}[\mathcal{W}]+C\nonumber
\end{alignat}

We first borrow the idea of checkpointing density from \textcite{Satoshi1992Optimal} and define an instantaneous sampling rate function $r(t)$ which is related to $\{t_n\}$ such that,
\begin{alignat}{1}
{\int_{t_{n-1}}^{t_n}}r(t)\dif t=1,\;\forall n\geq1.\label{rt}
\end{alignat}
Note that, for periodic sampling, this function is a constant, equal to the sampling frequency.
In the aperiodic case, we find $r^*(t)$, the $r(t)$ that minimizes $\mathcal{E}$.
By construction, the number of samples taken up to any time instant can be computed directly by computing the area under $r(t)$.
Thus, we obtain the expected number of samples $\mathbb{E}[\mathcal{S}]$ as
\begin{alignat}{1}
\mathbb{E}[\mathcal{S}]&=\int_{t=0}^{\infty}\bigg(\!\int_{x=0}^{t}\!\!\!\!r(x)\,\mathrm{d}x\bigg)f_{\mathcal{T}}(t)\,\mathrm{d}t.\label{Es}
\end{alignat}

To find 
% the expected wait 
$\mathbb{E}[\mathcal{W}]$, we use the conditional \ac{CDF} of the execution time.
\begin{multline*}
\mathbb{P}(\mathcal{W}=t_n-\mathcal{T}\leq t\,\big\vert\,t_{n-1}<\mathcal{T}\leq t_n)\\=\dfrac{\mathbb{P}(\mathcal{T}\geq t_n-t\,,\,t_{n-1}<\mathcal{T}\leq t_n)}{\mathbb{P}(t_{n-1}<\mathcal{T}\leq t_n)}.
\end{multline*}
The numerator is degenerate when $t\!<\!0$ or $t\!>\!(t_n\!-\!t_{n-1})$.
Thus, we are only interested in $0\!\leq\!t\!\leq\! (t_n-t_{n-1})$.
Let $F_{\mathcal{T}}$, $\bar{F}_{\mathcal{T}}$ and $f_{\mathcal{T}}$ correspond to the \ac{CDF}, \ac{CCDF} and \ac{PDF} of the execution time distribution.
\begin{alignat}{1}
\!\!\!\Rightarrow\mathbb{P}(\mathcal{W}\leq t\,\big\vert\,t_{n-1}<\mathcal{T}\leq t_n)&=\dfrac{\mathbb{P}(t_n-t\leq\mathcal{T}\leq t_n)}{F_\mathcal{T}(t_n)-F_\mathcal{T}(t_{n-1})}\nonumber\\
&\approx\dfrac{F_\mathcal{T}(t_n)-F_\mathcal{T}(t_{n}-t)}{F_\mathcal{T}(t_n)-F_\mathcal{T}(t_{n-1})}\label{Apx1}
\end{alignat}
Here, \cref{Apx1} is an approximation merely for mathematical maturity due to the slackness of the first inequality in the numerator.
Expanding the \ac{CDF} using Taylor series gives
\begin{multline*}
    =\Big(F_\mathcal{T}(t_n)-\\\big(F_\mathcal{T}(t_n)+f_\mathcal{T}(t_n)(-t)+f'_\mathcal{T}(t_n)(-t)^2/2!+\dots\big)\Big)\\
    \div \Big(F_\mathcal{T}(t_n)-\big(F_\mathcal{T}(t_n)+f_\mathcal{T}(t_n)(t_{n-1}-t_n)\\+f'_\mathcal{T}(t_n)(t_{n-1}-t_n)^2/2!+\dots\big)\Big).
\end{multline*}
Simplifying and using the first order approximation, we arrive at
\begin{alignat}{1}
% &=\dfrac{F_\mathcal{T}(t_n)-\big(F_\mathcal{T}(t_n)+f_\mathcal{T}(t_n)(-t)+f'_\mathcal{T}(t_n)(-t)^2/2!+\dots\big)}{F_\mathcal{T}(t_n)-\big(F_\mathcal{T}(t_n)+f_\mathcal{T}(t_n)(t_{n-1}-t_n)+f'_\mathcal{T}(t_n)(t_{n-1}-t_n)^2/2!+\dots\big)}\\&
\mathbb{P}(\mathcal{W}\leq t\,\big\vert\,t_{n-1}<\mathcal{T}\leq t_n)&\approx\dfrac{tf_\mathcal{T}(t_n)}{(t_{n}-t_{n-1})f_\mathcal{T}(t_n)}\label{Apx2}\\
\Rightarrow\mathbb{P}(\mathcal{W}> t\,|\,t_{n-1}<\mathcal{T}\leq t_n)&= 1-\dfrac{t}{(t_{n}-t_{n-1})}\nonumber
\end{alignat}

Next, using the above \ac{CCDF}, we find the conditional expectation of $\mathcal{W}$.
\begin{alignat*}{1}
% \mathbb{P}(\mathcal{W}> t\,|\,t_{n-1}<\mathcal{T}\leq t_n)&\approx 1-\dfrac{t}{(t_{n}-t_{n-1})}\\
\Rightarrow \mathbb{E}[\mathcal{W}\,|\,t_{n-1}<\mathcal{T}\leq t_n]&=\smashoperator[r]{\int_{0}^{t_n-t_{n-1}}}\Big(1-\dfrac{t}{(t_{n}-t_{n-1})}\Big)\,\mathrm{d}t\\
&=\dfrac{(t_n-t_{n-1})}{2}.
\end{alignat*}
% However, we can also approximate $(t_n-t_{n-1})$, the sampling interval to the inverse of the instantaneous sampling frequency $r(t)$. This approximation (\ref{Apx3}) is valid as long as $r(t)$ is varying slowly between two consecutive sampling instants. That is,
If $r(t)$ is varying slowly between two consecutive sampling instants, we can approximate the sampling interval $(t_n\!-\!t_{n-1})$ as
\begin{alignat}{1}
 (t_n-t_{n-1})&\approx\dfrac{1}{r(t)},\;\forall t,n:t_{n-1}\!<\!t\!\leq\!t_n.\label{Apx3}\\
% \end{alignat}
% As a result, we can compute the expected wait as,
% \begin{alignat}{1}
\Rightarrow \mathbb{E}[\mathcal{W}]&=\int_{0}^{\infty}\mathbb{E}[\mathcal{W}\,|\,t_{n-1}<\mathcal{T}\leq t_n]f_\mathcal{T}(t)\,\mathrm{d}t\nonumber\\
&=\int_{0}^{\infty}\dfrac{1}{2r(t)}f_\mathcal{T}(t)\,\mathrm{d}t.\label{Ew}
\end{alignat}
We can thus find the energy penalty using \cref{Es} and \cref{Ew} as
\begin{alignat*}{1}
\mathcal{E}&=\int_{0}^{\infty}\Big(\alpha\int_{x=0}^{t}r(x)\,\mathrm{d}x+\beta\dfrac{1}{2r(t)}\Big)f_{\mathcal{T}}(t)\,\mathrm{d}t.\label{epsilon_eulerForm}\\
\intertext{%
    Let $g(t)=\int_{0}^{t}r(x)\dif x$.
    Then  $g'(t)=\tfrac{\mathrm{d}}{\mathrm{d}t}g(t)=r(t)$.
    That is,
}
\mathcal{E}&=\int_{0}^{\infty}\Big(\alpha g(t)+\dfrac{\beta}{2g'(t)}\Big)f_{\mathcal{T}}(t)\,\mathrm{d}t
\end{alignat*}

As per the Euler-Lagrange equation from the calculus of variations~\cite{Bellman1954Dynamic,Arfken2015Calculus}, the extreme value of $\mathcal{E}$ is obtained at 
\begin{alignat}{1}
r^*(t)&=\sqrt{\dfrac{\beta f_\mathcal{T}(t)}{2\alpha\bar{F}_\mathcal{T}(t)}}.\\
\intertext{Thus, for a Rayleigh distributed $\mathcal{T}$ with parameter $\sigma$,}
r^*(t)&=\sqrt{\dfrac{\beta t}{2\alpha\sigma^2}}\\
\Rightarrow &\int_{t_n}^{t_{n+1}}\!\!\!\sqrt{\dfrac{\beta t}{2\alpha\sigma^2}}\,\mathrm{d}t=1,\;\forall n\geq1\nonumber\tag{from \eqref{rt}}\\
\Rightarrow &\;t_{n+1}^{\frac{3}{2}}-t_{n}^{\frac{3}{2}}=3\sigma\!\sqrt{\tfrac{\alpha}{2\beta}}\nonumber\\
\Rightarrow &\;t_n=\Big(3\sigma\!\sqrt{\tfrac{\alpha}{2\beta}}\Big)^{\frac{2}{3}}n^{\frac{2}{3}}\label{tnRayleigh}.
\end{alignat}

By making use of this general solution, we can also prove the results given in section \cref{ssec:optimization:samples}, where we modify the problem and find the optimum sampling instants that minimize the expected number of samples for a given upper bound $w_0$ for the expected wait time.
First note that the general solution has constants $\alpha$ and $\beta$ that correspond to the weights given to the cost of sampling and cost of waiting, respectively.
The optimization criteria changes from minimizing wait time to minimizing the number of samples when the ratio $\frac{\alpha}{\beta}$ goes from zero to infinity.
Since any positive real value is valid for this ratio, one can achieve any valid point $(\mathbb{E}[\mathcal{S}],\mathbb{E}[\mathcal{W}])$ via simply by varying the ratio $\frac{\alpha}{\beta}$.
Furthermore, since the sampling instants are aperiodic and can take any positive real values, the bound will be tight at the optimum.
Hence, to solve for the modified optimization problem explained in section \cref{ssec:optimization:samples} that minimizes $(\mathbb{E}[\mathcal{S}]$ with an upper bound $w_0$ on $\mathbb{E}[\mathcal{W}]$, we equate \cref{Ew} to $w_0$, find the corresponding $\frac{\alpha}{\beta}$, and find the optimum set of sampling instants by plugging this ratio into \cref{tnRayleigh}.
\begin{alignat*}{1}
\mathbb{E}[\mathcal{W}]&=\int_{0}^{\infty}\dfrac{1}{2r(t)}f_\mathcal{T}(t)\,\mathrm{d}t.\nonumber\\
&=\int_{0}^{\infty}\frac{1}{2}\sqrt{\frac{2\alpha\sigma^2}{\beta t}}\cdot \frac{t}{\sigma^2}e^{-{t^2}/{2\sigma^2}}\mathrm{d}t\\
&=\sqrt{\frac{\alpha\sigma^2}{2\beta}}\int_{0}^{\infty}\frac{\sqrt{t}}{\sigma^2}e^{-{t^2}/{2\sigma^2}}\mathrm{d}t\\
&=\sqrt{\frac{\alpha\sigma^2}{2\beta}}\left({\frac{1}{{2}\sigma^2}}\right)^{1/4}\cdot\int_{0}^{\infty}y^{-1/4}e^{-y}\mathrm{d}y\\
&=\sqrt{\frac{\alpha\sigma^2}{2\beta}}\left({\frac{1}{{2}\sigma^2}}\right)^{1/4}\cdot\mathlarger{\Gamma}(\tfrac{3}{4}),\\
% &\approx 0.728637\sqrt{\frac{\alpha\sigma}{\beta}}\\
\intertext{%
    where $\mathlarger{\Gamma(x)}$ is the gamma function.
    Thus, when the upper bound $w_0$ is tight,
}
\mathbb{E}[\mathcal{W}]&=w_0=\sqrt{\frac{\alpha\sigma}{2\sqrt{2}\beta}}\mathlarger{\Gamma}(\tfrac{3}{4})\\
\Rightarrow\frac{\alpha}{\beta}&=\frac{w_0^2}{\sigma}\frac{2\sqrt{2}}{(\mathlarger{\Gamma}(\tfrac{3}{4}))^2}\\
% &=1.88355\frac{w_0^2}{\sigma}
&\approx1.9\frac{w_0^2}{\sigma}
\end{alignat*}
