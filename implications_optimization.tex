\section{Implications for the study and optimization of \acs{WCA}}\label{sec:implications}
\todo[inline]{Implications for task durations, wrt to naive scheme.}
\todo[inline]{Implications for optimization in terms of sampling, energy.}
\todo[inline]{Introduction to this section}

We begin by studying the implications of such a model on the estimation of application lifetimes.
In the context of \ac{WCA}, we will understand \emph{application lifetime} as the time it takes a user to complete a specified task.
This is an important metric for \ac{WCA} optimization, as it directly relates to system resource utilization and contention, and to energy consumption.

In order to illustrate the consequences of using a less realistic, less accurate model for this kind of research, we introduce here a reference model to which we will compare our results.
This reference model represents one of the most crude possible approximations the empirical execution time modeling.
It consist simply of an \ac{exGaussian} distribution fitted to all execution time samples collected for \textcite{olguinmunoz:impact2021} which is randomly sampled at runtime without any sort of adjustment to the current state of the system.

We start by studying application lifetimes in a controlled setup.
