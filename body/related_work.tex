\section{Related work}\label{sec:relwork}

\todo[inline]{Expand}

There exist a number of approaches to the characterization and benchmarking of mobile \ac{XR}, and by extension \ac{WCA}, deployments.
\emph{OpenRTiST}~\cite{george2020openrtist} is a tool which uses a compute-heavy, yet latency-sensitive workload --- image style transfer~\cite{jing2019neural} --- to load and benchmark edge computing deployments.
It acts as an end-to-end realistic \ac{AR} workload, allowing thus researchers to measure the real end-to-end latencies of such systems.
\textcite{chen2017empirical} use a number of prototype \ac{WCA} implementation to study and characterize real-world latency bounds in these systems.
In~\cite{lecci2021open}, the authors collect and analyze large amounts of \ac{VR} network traffic, which they then use to construct a synthetic model for the generation of traces of such traffic.
\textcite{chetoui2022arbench} propose \emph{ARBench}, a toolkit for the benchmarking of mobile hardware in the context of \ac{AR}.
The toolkit incorporates a series of workloads which stress different components of the mobile device and calculates a score for each workload.
\emph{Yahoo Cloud Serving Benchmark}~\cite{cooper2010benchmarking} and \emph{DCBench}~\cite{jia2013characterizing} are workload sets intended for benchmarking cloud services which have further been used in the context of benchmarking edge computing infrastructure for \ac{AR} suitability.
\emph{Edgebench}~\cite{das2018edgebench} and \emph{Defog}~\cite{mcchesney2019defog} benchmark workload performance on the edge versus on the cloud.

We contribute to this body of work by providing the first tool for the benchmarking of \ac{WCA} that dynamically considers the effects of human behavior.
In contrast to the above described works, which generate loads with relatively static profiles, our model is able to react to changes in system responsiveness like a human would.
We achieve this by building upon our own previous contributions to this field.
In~\cite{olguinmunoz2018demoscaling,olguinmunoz2019edgedroid} we introduced a coarse approximation to human behavior modeling in \ac{WCA} we called \emph{EdgeDroid}.
We used a trace-driven approach, where a pre-recorded and pre-processed ``ideal'' trace of steps for a specific \ac{WCA} step-based task is replayed to a Gabriel~\cite{chen2018application} backend.
In order to adapt to potential mismatches between system responsiveness at trace-capture time and trace-replay time, we used a simple \ac{FSM} to adapt the trace at the latter by replaying or skipping certain segments.
In~\cite{olguinmunoz2021impact} we studied the effects of reduced system responsiveness on human behavior in \ac{WCA} applications through human-subject studies.
We found that humans generally pace themselves according to the perceived system responsiveness.
When interacting with a highly responsive system, humans tended to speed up with each step; conversely, humans tended to slow down in highly unresponsive states.