\documentclass[10pt, letterpaper]{letter}
\usepackage{hyperref}
\usepackage[T1]{fontenc}
\usepackage[utf8]{inputenc}
\usepackage[]{todonotes}
\usepackage[]{geometry}
\usepackage[inline]{enumitem}
\usepackage[nolist]{acronym}
\usepackage[]{siunitx}
\usepackage[
  style=numeric-comp,
  sorting=none,
  sortcites,
  hyperref,
  mincitenames=1,
  maxcitenames=2,
  maxbibnames=2,
  minbibnames=1,
  citestyle=numeric-comp, % for [1, 2] instead of [1], [2]
  backend=bibtex
]{biblatex}
\bibliography{bibliography.bib}
% \AtBeginBibliography{\small}
% \AtEveryBibitem{\clearfield{day}}
% \AtEveryBibitem{\clearfield{isbn}}
% % \AtEveryBibitem{\clearfield{url}}
% \AtEveryBibitem{\clearfield{series}}
% \AtEveryBibitem{\clearlist{location}}
% \AtEveryBibitem{\clearfield{doi}}
\begin{acronym}
    \acro{CPS}{Cyber-Physical System}
    \acro{NCS}{Networked Control System}
    \acroindefinite{NCS}{an}{a}
    \acro{LAN}{Local-Area Network}
    \acro{WLAN}{Wireless Local-Area Network}
    \acro{KPI}{Key Performance Indicator}
    \acro{WPAN}{Wireless Personal Area Network}
    \acro{CSMA/CD}{Carrier-Sense Multiple Access with Collision Detection}
    \acro{CLEAVE}{ControL bEnchmArking serVice on the Edge}
    \acro{OS}{Operating System}
    \acro{UDP}{User Datagram Protocol}
    \acro{TCP}{Transmission Control Protocol}
    \acro{RMS}{Root Mean Square}
    \acro{RTT}{Round-Trip Time}
    \acro{CI}{Confidence Interval}
    \acro{AP}{Access Point}
    \acro{API}{Application Programming Interface}
    \acro{SSF}{Swedish Foundation for Strategic Research}
    \acro{TECoSA}{Trustworthy Edge Computing Systems and Applications}
    \acro{ABC}{Abstract Base Class}
    \acroplural{ABC}[ABCs]{Abstract Base Classes}
    \acro{URL}{Uniform Record Locator}
    \acro{AWS}{Amazon Web Services}
    \acro{EC2}{Elastic Compute 2}
    \acro{AMI}{Amazon Machine Image}
    \acro{SSH}{Secure Shell}
    \acro{IP}{Internet Protocol}
    \acro{VPN}{Virtual Private Network}
    \acro{NAT}{Network Address Translation}
    \acro{NTP}{Network Time Protocol}
    \acroindefinite{NTP}{an}{a}
    \acro{SDR}{Software-Defined Radio}
    \acroindefinite{SDR}{an}{a}
    \acro{VLAN}{Virtual Local-Area Network}
    \acro{APT}{Advaced Packaging Tool}
    \acro{LTE}{Long-Term Evolution}
    \acro{SSID}{Service Set Identifier}
    \acro{EPC}{Evolved Packet Core}
    \acro{UE}{User Equipment}
    \acro{eNodeB}{Evolved Node B}
    \acro{DNS}{Domain Name System}
    \acro{MNIST}{Modified National Institute of Standards and Technology}
    \acro{HTTP}{Hyper-Text Transfer Protocol}
    \acroindefinite{HTTP}{an}{a}
    \acro{YAML}{YAML Ain't Markup Language}
    \acro{O-RAN}{Open Radio Access Network}
    \acro{FOSS}{Free and Open Source Software}
    \acro{COSMOS}{Cloud enhanced Open Software defined MObile wireless testbed for city-Scale deployment}
    \acro{OMF}{ORBIT Management Framework}
    \acro{POWDER}{Platform for Open Wireless Data-driven Experimental Research}
    \acro{COTS}{Commercial Off-The-Shelves}
    \acro{RF}{Radio-Frequency}
    \acro{VM}{Virtual Machine}
    \acro{REST}{REpresentational Sate Transfer}
    \acro{OS}{Operating System}
    \acro{MaaS}{Metal-as-a-Service}
    \acro{OEDL}{\ac{OMF} Experiment Description Language}
    \acro{RSpec}{Resource Specification}
    \acro{LXC}{LinuX Containers}
    \acro{MEC}{Mobile Edge Computing}
    \acro{USRP}{Universal Software Radio Peripheral}
    \acro{OAI}{OpenAirInterface}
    \acro{NR}{New Radio}
    \acro{FPGA}{Field-Programmable Gate Array}
    \acro{WCA}{Wearable Cognitive Assistance}
    \acro{TTF}{Time-to-Feedback}
    \acro{AR}{Augmented Reality}
    \acro{VR}{Virtual Reality}
    \acro{XR}{eXtended Reality}
    \acro{MAR}{Mobile \ac{AR}}
    \acro{GPS}{Global Positioning System}
    \acro{QoE}{Quality of Experience}
    \acro{QoS}{Quality of Service}
    \acro{TTF}{Time-to-Feedback}
    \acro{PCA}{Principal Component Analysis}
    \acro{exGaussian}[Ex-Gaussian]{Exponentially Modified Gaussian}
    \acro{IEEE}{Institute of Electrical and Electronics Engineers}
    \acro{CPU}{Central Processing Unit}
    \acro{N/A}{Not Applicable}
    \acro{CIO}{Confidence Interval}
    \acro{MLE}{Maximum Likelihood Estimation}
    \acro{CDF}{Cumulative Distribution Function}
    \acro{CCDF}{Complementary \ac{CDF}}
    \acro{PDF}{Probability Density Function}
    \acro{XR}{eXtended Reality}
    \acro{FSM}{Finite State Machine}
    \acro{ITQ}{Immersive Tendencies Questionnaire}
    \acro{BFI}{Big Five Inventory of personality traits}
    \acro{PCA}{Principal Component Analysis}
    \acro{ECDF}{Empirical \ac{CDF}}
    \acro{NSF}{United States National Science Foundation}
\end{acronym}



\signature{
    Manuel {Olguín Muñoz}\\
    Division of Information~Science and Engineering, EECS\\
    KTH Royal Institute of Technology\\
    Malvinas väg 10, 7th floor\\
    100 44 Stockholm, Sweden\\
    \href{mailto:molguin@kth.se}{\url{molguin@kth.se}}
}
% \address{}
\begin{document}

\begin{letter}{
    Qian Zhang, Ph.D.\\
    Editor-in-Chief\\
    IEEE Transactions on Mobile Computing
}
\opening{Dear Dr.\ Zhang:}

Please find enclosed a research article submission to IEEE Transactions on Mobile computing entitled \emph{``Realistic Modeling of Human Timings for Wearable Cognitive Assistance''}.
% We confirm that this work is original and has not been published elsewhere, nor is it currently under consideration for publication elsewhere.

This manuscript presents two main contributions.
We first introduce a stochastic model for human behavior in the context of \ac{WCA}.
This model is built from data collected for our previous work in ``\citefield{olguinmunoz:impact2021}{title}'' (\citeauthor{olguinmunoz:impact2021}, \citefield{olguinmunoz:impact2021}{year}), and represents the first realistic end-to-end modeling approach for \ac{WCA}.
Given the difficulty of scalably and repeatably studying systems such as \ac{WCA} which include a human-in-the-loop, we believe this model to be an important contribution to the development and expansion of this field of research.

Furthermore, we introduce a novel optimization framework for resource consumption-responsiveness trade-offs in these systems.
We show that by combining these two contributions, we are able to obtain significant reductions --- up to \SI{50}{\percent} --- in relevant metrics relating to system load and resource consumption, such as number of processed samples and raw energy consumption per logical application step.
We believe these results provide key insights for developers and researchers working on developing and benchmarking \ac{WCA} applications as well as on cost-efficient strategies for resource allocation in these systems.

We do not have any opposed reviewers or editors.

I will be serving as the corresponding author for the manuscript, assuming all the responsibilities this entails. 
All of the authors listed have agreed to the byline order and to submission of the manuscript, and understand that, if accepted for publication, a certification of authorship form will be required that will be signed by all of us.

\closing{Sincerely,}
\end{letter}
\end{document}