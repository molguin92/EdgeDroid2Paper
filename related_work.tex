\section{Related work}\label{sec:relwork}

Our contributions build on previous work on the characterization of human perception of delay in \ac{WCA} systems.
We expand upon the findings of \textcite{Ha2014towards,Chen2017Empirical}, who identified the need and opportunities for edge offloading of \ac{WCA} and characterized latency bounds in these systems.

Furthermore, we directly build upon and expand on our previous work on the modeling of human behavior and characterization of human timings in \ac{WCA}, as well as on optimum sampling of edge-bound interactive systems.
In \textcite{olguin2018scaling,olguin2019edgedroid} we introduced a coarse approximation to human behavior modeling in \ac{WCA} we called \emph{EdgeDroid}.
We used a trace-driven approach, where a pre-recorded and pre-processed ``ideal'' trace of steps for a specific \ac{WCA} step-based task is replayed to a Gabriel~\cite{Chen2018application} backend.
In order to adapt to potential mismatches between system responsiveness at trace-capture time and trace-replay time, we used a simple \ac{FSM} to adapt the trace at the latter by replaying or skipping certain segments.
In \textcite{olguinmunoz:impact2021} we studied the effects of reduced system responsiveness on human behavior in \ac{WCA} applications through human-subject studies.
We found that humans generally pace themselves according to the perceived system responsiveness.
When interacting with a highly responsive system, humans tended to speed up with each step; conversely, humans tended to slow down in highly unresponsive states.
We name the new model presented in this paper \emph{\edgedroid}; a direct, more realistic evolution of our initial EdgeDroid approach.
Finally, in \textcite{Moothedath2021EnergyOptimal,Moothedath2022EnergyEfficient} we find the optimum periodic sampling interval which minimizes the energy tracking human progress of a specific subtask in \ac{WCA}.
This is further extended into an aperiodic strategy in \textcite{Moothedath2022Aperiodic}.

\todo[inline]{%
Include Zorzi et al. AR VR
}